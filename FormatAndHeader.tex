% LaTeX Template für Abgaben an der Universität Stuttgart
% Autor: Sandro Speth
% Bei Fragen: Sandro.Speth@studi.informatik.uni-stuttgart.de
%-----------------------------------------------------------
% Modul beinhaltet Befehl fuer Aufgabennummerierung,
% sowie die Header Informationen.

% Überschreibt enumerate Befehl, sodass 1. Ebene Items mit
\renewcommand{\theenumi}{(\alph{enumi})}
% (a), (b), etc. nummeriert werden.
\renewcommand{\labelenumi}{\text{\theenumi}}

% Counter für das Blatt und die Aufgabennummer.
% Ersetze die Nummer des Übungsblattes und die Nummer der Aufgabe
% den Anforderungen entsprechend.
% Gesetz werden die counter in der hauptdatei, damit siese hier nicht jedes mal verändert werden muss
% Beachte:
% \setcounter{countername}{number}: Legt den Wert des Counters fest
% \stepcounter{countername}: Erhöht den Wert des Counters um 1.
\newcounter{sheetnr}
\newcounter{exnum}

% Befehl für die Aufgabentitel
\newcommand{\exercise}[1]{\section*{Aufgabe \theexnum\stepcounter{exnum}: #1}} % Befehl für Aufgabentitel

% Informationen
\newcommand{\firstnameA}{Max}
\newcommand{\firstnameB}{Klaus}
\newcommand{\lastnameA}{Mustermann}
\newcommand{\lastnameB}{Kleber}
\newcommand{\registrationNumberA}{1234567}
\newcommand{\registrationNumberB}{1234568}

% Befehl für das Deckblatt
\newcommand{\modTitle}{\section*{Modellierung Sommersemester 2018}
Aufgabenblatt \thesheetnr \vspace{0.5cm} \\
\begin{tabular}{ | l | l | l | }
 \hline
 Name & Vorname & Matrikelnummer \\
 \hline
 \lastnameA&\firstnameA&\registrationNumberA\\
 \hline
 \lastnameB&\firstnameB&\registrationNumberB\\
 \hline
\end{tabular} \vspace{0.5cm} \\
Die Bearbeitung der Aufgabenblätter muss durch zwei in Ilias registrierte Mitglieder des Kurses „Modellierung (SS18)“ erfolgen. \\
\vspace{3cm}\\
In der folgenden Tabelle werden die erzielten Punkte eingetragen. \vspace{0.5cm} \\
\begin{tabular}{ | c | c | c | }
 \hline
 Aufgabe & Erreichte Punkte & Bemerkung zur Korrektur \hspace{5cm} \\
 \hline
 1&&\\[8pt]
 \hline
 2&&\\[8pt]
 \hline
 3&&\\[8pt]
 \hline
 4&&\\[8pt]
 \hline
 5&&\\[8pt]
 \hline
 6&&\\[8pt]
 \hline
 7&&\\[8pt]
 \hline
 8&&\\[8pt]
 \hline
 Gesamt&&\\[8pt]
 \hline
\end{tabular}
\newpage
}

% Formatierung der Kopfzeile
% \ohead: Setzt rechten Teil der Kopfzeile mit
% Namen und Matrikelnummern aller Bearbeiter
\ohead{\firstnameA \lastnameA (\registrationNumberA)\\
\firstnameB \lastnameB (\registrationNumberB)}
% \chead{} kann mittleren Kopfzeilen Teil sezten
% \ihead: Setzt linken Teil der Kopfzeile mit
% Modulnamen, Semester und Übungsblattnummer
\ihead{Modellierung\\
Sommersemester 2018\\
Übungsblatt \thesheetnr}